\documentclass[a4paper]{article}

\usepackage[margin=3.5cm]{geometry}
\usepackage{booktabs,amsmath,paralist}

\begin{document}

\title{
  {\small
    Halmstad University course DA4002\\
    Introduction to Algorithms, Data Structures, and Problem Solving\\
  }
  Group Activities for Lecture 6
}
\maketitle



\section{Dynamic Programming applied to the coin change problem}

a country has coin denominations of 1, 3, and 7 cents
what's the best way to give change?
minimize number of coins
give change: the coins must sum to K
do it for K = { 1, ..., 10 }



\section{Equipment replacement}

\begin{table}
  \caption{Operation and investment costs and values.}\label{tab:costs}
  \centering
  \begin{tabular}{lrrrr}
    \toprule
    \multicolumn{2}{l}{Operation or investment} & year 1    & year 2    & year 3    \\
    \midrule
    purchase cost                     & $p=1000$ &           &           &           \\
    sales value                       &          & $s_1=800$ & $s_2=600$ & $s_3=500$ \\
    maintenace cost                   &          & $m_1=60$  & $m_2=80$  & $m_3=120$ \\
    \bottomrule
  \end{tabular}
\end{table}

The class is divided into two ``random'' groups.
Each group is in charge of a production facility that needs a certain type of machine.
A machine can be kept for a maximum of three years.
The value that you get for selling an old machine, and the cost of maintaining a machine, both depend on its age.
A new machine always costs the same amount.
What is the cheapest sequence of buying, maintaining, and selling such machines over the next five years, given the costs and values from table~\ref{tab:costs}?

Use the following hints to formulate and solve this problem.
Discuss with your neighbor or in the entire group to make sure you understand what is meant.

\begin{compactitem}
\item
  The natural \emph{stages} $n$ (or steps) in this problem are the years of operation, and it is easiest to plan backwards.
\item
  The natural \emph{state} $x$ in this problem is the age of the machine expressed as year of operation.
  Thus, if you just bought a machine, its age will be 1 (one) and it will cost you $m_1$ to maintain.
  Or, if you sell a 2 year old machine, you will receive $s_2$ money back.
\item
  There are two possible \emph{actions} $a$ which encode what you can do at the beginning of each year:
  either \emph{keep} a machine, or \emph{trade} it in for a new one.
  But you do not always have the choice, for example after using a 3 year old machine, you have to trade.
\item
  Think about how to formalize the \emph{transformation} $x_{n+1}=T(x_n,a_n)$ and the \emph{cost} $F(x_n,a_n)$ associated with each action.
  To shorten notation, we can write $a=\text{K}$ to denote \emph{keep} actions, and $a=\text{T}$ for \emph{trade} actions.
\item
  Express the \emph{value function} update (Bellman equation) in terms of the problem-specific formalisations.
  In this problem, we are minimizing cost (as opposed to maximizing payoff).
\item
  After preparing all the above ingredients, begin by thinking about what are the possible outcomes of the last (fifth) year in isolation.
  Then, for each of these outcomes, what would be the possible and the best options for year four?
  And so on until you reach year zero.
\end{compactitem}





\end{document}
