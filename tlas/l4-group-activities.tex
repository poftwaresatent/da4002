\documentclass[a4paper]{article}

\usepackage[margin=3cm]{geometry}
\usepackage{verbatim,amsmath}

\begin{document}

\title{
  {\small
    Halmstad University course DA4002\\
    Introduction to Algorithms, Data Structures, and Problem Solving\\
  }
  Group Activities for Lecture 4
}
\maketitle



\section{Stack and Queue Implementation}

This is a pen-and-paper design and implementation activity emphasizing skills didactic and teamwork skills.

Students are alternatingly assigned to groups A and B.
Everyone receives a printout of source code for vectors and singly linked lists.
Each student then individually and independently creates an implementation of either a stack (group A) or a queue (group B) according to a specified interface.
This should take only a couple of minutes.

Then, each student from group A explains their implementation to a student from group B.
Explain what you have done and why.
Draw sketches or diagrams to clearly illustrate the most important points.
After a few minutes, reverse the roles:
group B students now explain their implementation to group A students.

Now that the stack and queue implementations are clarified, the pairs of students work together to design and sketch a test program that can be used to verify that the data structure implementations are correct.
It is important to focus not only on what the various operations should do, but also on things that could go wrong in case of incorrect implementations.
This should also take only a couple of minutes.

\emph{The results of the exercise are not handed in.
  To check whether the implementations are correct, the students can later write the test program that they designed.}



\section{Working with Big-Oh Expressions}

Together with another student (for example the same as for the previous activity), discuss and write down the ``Big-Oh'' complexity expression $O(N)$ for the following execution times $T(N)$.

\begin{align}
  T(N) &= 0.15 N^2 + 23.5 \log N
  \\
  T(N) &= N (5N + (22 + 0.01N^2) \log N) + 2.1 \cdot 2^N
  \\
  T(N) &= \sum_{i=1}^Ni
  \\
  T(N) &=
  \begin{cases}
    0          & \text{for} N < 1 \\
    N + T(N-1) & \text{otherwise}
  \end{cases}
  \\
  T(N) &=
  \begin{cases}
    1        & \text{for} N \leq 1 \\
    2 T(N-1) & \text{otherwise}
  \end{cases}
\end{align}

\end{document}
