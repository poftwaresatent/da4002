\documentclass[a4paper]{article}

\usepackage{amsmath}
\usepackage[pdftex]{graphicx}
\usepackage[margin=3.5cm]{geometry}

\begin{document}

\title{
  {\small
    Halmstad University course DA4002\\
    Introduction to Algorithms, Data Structures, and Problem Solving\\
  }
  Group Activities for Lecture 7
}
\maketitle



\section{Graph Representations}

In groups of two students (with your neighbor), work on the following questions.
Figure~1 shows four example graphs labeled \textbf{G1}, \textbf{G2}, \textbf{G3}, and \textbf{G4}.
Figure~2 shows two possible ways of representing graphs, \textbf{D1} and \textbf{D2}.

\begin{enumerate}
\item
  Which of the graphs \textbf{G1}, \textbf{G2}, \textbf{G3}, or \textbf{G4} is represented by \textbf{D1}?
\item
  And which one is represented by \textbf{D2}?
\item
  For the two graphs that are not represented by either \textbf{D1} or \textbf{D2}, make a sketch of how it would be represented as an adjacency list and as an adjacency matrix.
\item
  Discuss the trade-off between adjacency lists and adjacency matrices with your neighbor:
  when would you use one, and when the other?
\end{enumerate}

\noindent
\emph{This group activity is based on question~3 of the written exam from January, 2012.}



\section{Graph Traversals}

In groups of two students (with your neighbor), work on the following questions.
Figure~3 shows a graph and its adjaceny matrix representation.
This defines the iteration order for the edges, which is an important detail here:
the order is given by reading from left to right along the appropriate row of the adjacency matrix.
For example, the order of edges emanating from vertex \textbf{C} is: first \textbf{A}, then \textbf{B}, and finally \textbf{F}.

Listing~1 shows pseudo-code for breadth-first search (BFS) and Listing~2 for depth-first search (DFS), the two fundamental graph traversal methods.

\begin{enumerate}
\item
  Perform BFS starting at vertex \textbf{D}.
  Write down the order in which the vertices are visited.
\item
  Perform DFS, starting at vertex \textbf{A}, also writing down the visitation order.
  Note that you will need to keep track of the function call stack, because DFS is a recursive algorithm.
\end{enumerate}

\noindent
\emph{This group activity is based on question~4 of the written exam from January, 2012.}


\pagebreak
\section{Topological Ordering}

The class is divided in to two larger groups to work on the following questions.

\begin{enumerate}
\item
  Each student first reads the following definition and theorem.
  Then the meaning of the definition and theorem are discussed in the whole group to make sure everyone understands.
  \begin{description}
  \item[Definition:]
    A \textbf{topological ordering} of a directed graph is a sequence of its vertices such that, for every edge $(u,v)$, $u$ appears before $v$ in the sequence.
    Note that for any path from $u$ to $v$, this also implies that $u$ comes before $v$ in the ordering.
    Figure~4 shows an example graph where the sequence $(C,A,D,B)$ is a topological ordering.
  \item[Theorem:]
    A topological ordering is possible if and only if the graph has no cycles.
    In other words, a graph has to be a \textbf{directed acyclic graph} (DAG) in order to have a topological ordering.
    And conversely, every DAG has at least one topological ordering.
  \end{description}
\item
  As a group, discuss to find out how you can prove the theorem.
\item
  Each student individually applies the algorithm shown in Listing~3 to the two graphs in Figure~5.
  This algorithm computes a topological order for a given directed graph.
  Compare your results with someone else in your group: which graph contains a cycle, and where is that cycle?
\end{enumerate}

\noindent
\emph{This group activity is based on question~6 of the written exam from October, 2011.}

\end{document}
