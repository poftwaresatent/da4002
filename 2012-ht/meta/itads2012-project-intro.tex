\documentclass[a4paper,10pt]{article}

\usepackage[margin=3cm]{geometry}

\begin{document}

\title{
  {\normalsize
    Introduction to Algorithms, Data Structures, and Problem Solving\\
    DA4002 (HT12) Halmstad University}\\
  Introduction to Projects
}
\author{
  \texttt{roland.philippsen@hh.se}
}
\maketitle

The ITADS course presents fundamental concepts and basic implementations for frequently used data structures and algorithms.
The lecture is complemented by hands-on exercises and projects, which provide a practical and empirical understanding of the theory.
The lecturer (or a knowledeable substitute) is available in the \textsc{Unix} room B231c during specific hours in order to help participants work on their exercises and projects.

Exercises are relativey small programming tasks, they are not part of student evaluation.
Projects are bigger tasks, require more independent work, and are performed in teams (two participants).
Working on the projects is a hands-on problem solving experience.
Participants are expected to apply not only the technical skills practised during exercises, but also use the course books and online resources to solve the problems encountered during project work.
For example, the lecture handouts and the course website cite several Wikipedia articles with explanations and pseudo-code for sorting algorithms.

Each project is evaluated and the obtained grade flows into the overall evaluation of each participant.
Each project counts for 2.5 credits, together with the written exam (also 2.5 credits) to total for the course is thus 7.5 credits.
A 25-point system is used in order to be compatible both with the Halmstad University grades and the ECTS grades.
The table~\ref{tab:grades} gives the conversion rules.

\begin{table}
  \centering
  \small
  \noindent
  \begin{tabular}{|l|c|c|c|c|c|c|c|c|c|c|c|c|c|c|c|c|}
    \hline
    points & $\leq$10 & 11 & 12 & 13 & 14 & 15 & 16 & 17 & 18 & 19 & 20 & 21 & 22 & 23 & 24 & 25 \\
    \hline
    HH grade & U & \multicolumn{5}{|c|}{3} & \multicolumn{5}{|c|}{4} & \multicolumn{5}{|c|}{5} \\
    \hline
    ECTS grade & F & \multicolumn{3}{|c|}{E} & \multicolumn{3}{|c|}{D} & \multicolumn{3}{|c|}{C} & \multicolumn{3}{|c|}{B} & \multicolumn{3}{|c|}{A} \\
    \hline
  \end{tabular}
  \caption{
    Conversion from 25-point system to HH and ECTS grades.
  }\label{tab:grades}
\end{table}

Under normal circumstances, both members of a team receive the same grade for a given project.
Exceptions can be made in case of major disagreements within the team.
The lecturer must be notified as early as possible of such circumstances, and a solution will be determined on a case-by-case basis.

Project evaluation is based on the following aspects:

\begin{description}

\item[Time constraints:]
  there is a strict deadline for handing in the source code and the report.
  Teams who miss the deadline will receive a penalty of 5 points.
  In case of exonerating circumstances, such as sickness certified by a medical doctor, a deadline extension will of course be granted.
  Participants must notify the lecturer of such circumstances as soon as possible when they arise.
  
\item[Technical correctness:]
  in principle, all source code must compile and execute without error on the workstations the \textsc{Unix} room B231c.
  Exceptions to this rules are possible in order to allow inclusion of partially finished code.
  Such code must be very clearly marked, and the report must explain the state of development and discuss the work which remains before the code would function properly.
  For example, if the team was working on an advanced sorting algorithm for extra points, but they were not able to finish before the deadline, the partially finished algorithm may give some extra points.
  
\item[Documentation:]
  a (short!) written report is an essential part of the project.
  It must follow a specific format and clearly present the work performed by the team.
  In terms of project evaluation, the most important aspects of the report are:
  \begin{itemize}
  \item Present what has been done by the team during the project.
  \item Explain how to run the code developed by the team.
  \item Clearly present the obtained technical results.
  \end{itemize}
  A separate document gives detailed guidelines for writing a project report.
  The ITADS course focuses on technical aspects of programming and problem solving.
  Reports must be written in English, but language quality is secondary: as long as it remains comprehensible, English errors are not relevant.
  Verbatim quotes (copy-pasting) from other sources is acceptable \textbf{if they are short, properly marked, and adequately cited} (the separate report writing guideline gives an example).
  If language is a major obstacle for a team, it is possible to receive a deadline extension for the report.
  Notify the lecturer as early as possible in order to arrange such an extension.
  
  Note that Halmstad University offers an \textbf{English Language Studio} for students who seek help with written English.
  The studio is available on Wednesdays between 9:00 and 12:00 in room F-208.
  The teacher is Nicholas Lloyd-Pugh \texttt{<nicholas.lloyd-pugh@hh.se>} (phone 167372).
  
\end{description}

\end{document}
