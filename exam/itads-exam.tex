\documentclass[a4paper]{article}

\usepackage[pdftex]{graphicx}
\usepackage[margin=3cm]{geometry}
\usepackage{verbatim,url}


\newcounter{question}
\newcommand{\question}{\refstepcounter{question}\section*{Question~\thequestion}}
\renewcommand*\thequestion{\arabic{question}}


\begin{document}

\pagestyle{empty}
\thispagestyle{empty}



\noindent
\begin{minipage}{\columnwidth}
  \centering
  \Large
  DA4002 (HT11) Halmstad University\\
  Introduction to Algorithms, Data Structures, and Problem Solving\\[3\baselineskip]
  \Huge
  Written Exam\\
  \Large
  Monday, October 24, 2011\\[2\baselineskip]
  Examiner: Roland Philippsen (phone 7249)
\end{minipage}

\vfill



\section*{Rules}

Aside from the obvious rules of conduct exams (e.g.\ no chatting):

\begin{itemize}
\item
  \textbf{No computing devices} (laptops, phones, calculators, \emph{etc}).
\item
  \textbf{No books or printouts}.
\item
  \textbf{Allowed self-written notes}: two sheets of A4 paper (front and back).
\end{itemize}



\section*{General Guidelines}

\begin{itemize}
\item
  \textbf{Read carefully} and pace yourself.
  You can solve the problems in any order you want, but later problems may be easier to solve after you have answered the preceding questions.
\item
  \textbf{Write clearly} and draw clear diagrams.
  Clearly indicate the question number for each of your answers.
  If you need to correct a mistake, then cleanly cross out the wrong answer and clearly indicate where the correction can be found.
\end{itemize}



\pagebreak
\pagestyle{plain}
\thispagestyle{plain}
\setcounter{page}{1}



\question\label{q:container-structures-a}

The following figure shows four examples of data structures.
They are labelled with the letters \textbf{(A)}, \textbf{(B)}, \textbf{(C)}, and \textbf{(D)}.
The table below the diagrams lists names of data structure types.

For each type listed in the table, determine whether it is represented by one of the diagrams in the figure.
Write the corresponding diagram letter into the second column of the table, or write a big \textbf{X} if there is no diagram for the given type.

\fbox{make sketches of list, k-ary tree, undirected graph, and directed \textbf{cyclic} graph.}

\begin{center}
  \begin{tabular}{|l|p{0.3\columnwidth}|}
    \hline
    \emph{container type} & \emph{letter from the above figure} \\
    \hline
    & \\
    list & \\
    & \\
    \hline
    & \\
    k-ary tree & \\
    & \\
    \hline
    & \\
    binary tree & \\
    & \\
    \hline
    & \\
    undirected graph & \\
    & \\
    \hline
    & \\
    directed acyclic graph & \\
    & \\
    \hline
  \end{tabular}
\end{center}

\clearpage

\question
\label{q:container-structures-b}
In question~\ref{q:container-structures-a}, two of the types listed in the table are missing from figure.
Draw a diagram for each of them, clearly indicating which diagram is for which container type.

\end{document}
