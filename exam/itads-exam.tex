\documentclass[a4paper]{article}

\usepackage[pdftex]{graphicx}
\usepackage[margin=3cm]{geometry}
\usepackage{verbatim,url}


\newcounter{question}
\newcommand{\question}{\refstepcounter{question}\section*{Question~\thequestion}}
\renewcommand*\thequestion{\arabic{question}}


\begin{document}

\pagestyle{empty}
\thispagestyle{empty}



\noindent
\begin{minipage}{\columnwidth}
  \centering
  \Large
  DA4002 (HT11) Halmstad University\\
  Introduction to Algorithms, Data Structures, and Problem Solving\\[3\baselineskip]
  \Huge
  Written Exam\\
  \Large
  Monday, October 24, 2011\\[2\baselineskip]
  Examiner: Roland Philippsen (phone 7249)
\end{minipage}

\vfill



\section*{Rules}

Aside from the obvious rules of conduct exams (e.g.\ no chatting):

\begin{itemize}
\item
  \textbf{No computing devices} (laptops, phones, calculators, \emph{etc}).
\item
  \textbf{No books or printouts}.
\item
  \textbf{Allowed self-written notes}: two sheets of A4 paper (front and back).
\end{itemize}



\section*{General Guidelines}

\begin{itemize}
\item
  \textbf{Read carefully} and pace yourself.
  You can solve the problems in any order you want, but later problems may be easier to solve after you have answered the preceding questions.
\item
  \textbf{Write clearly} and draw clear diagrams.
  Clearly indicate the question number for each of your answers.
  If you need to correct a mistake, then cleanly cross out the wrong answer and clearly indicate where the correction can be found.
\end{itemize}



\pagebreak
\pagestyle{plain}
\thispagestyle{plain}
\setcounter{page}{1}



\question\label{q:container-structures-a}

The following four figures show four examples of data structures.
They are labelled with the letters \textbf{(A)}, \textbf{(B)}, \textbf{(C)}, and \textbf{(D)}.
The table below them lists names of data structure types.
For each type listed in the table, determine whether it is represented by one of the figures.
If there is a figure for a given type, write the corresponding letter into the second column of the table.
Mark structure types which are not shown in any diagram with a big \textbf{X}.\\[\baselineskip]

\noindent
\begin{minipage}[b]{0.44\columnwidth}
  \fbox{\textbf{(A)}\includegraphics[width=\columnwidth,trim=1cm 1cm 1cm 1cm]{fig/k-ary-tree.pdf}}
  \fbox{\textbf{(B)}\includegraphics[width=\columnwidth,trim=1cm 1cm 1cm 1cm]{fig/simply-linked-list.pdf}}
\end{minipage}
\hfill
\begin{minipage}[b]{0.5\columnwidth}
    \fbox{\textbf{(C)}\includegraphics[width=\columnwidth,trim=1cm 1cm 1cm 1cm]{fig/directed-cyclic-graph.pdf}}
    \fbox{\textbf{(D)}\includegraphics[width=\columnwidth,trim=1cm 1cm 1cm 1cm]{fig/undirected-graph.pdf}}
\end{minipage}

\begin{center}
  \begin{tabular}{|l|p{0.3\columnwidth}|}
    \hline
    \emph{container type} & \emph{A, B, C, D, or X} \\
    \hline
    & \\
    simply linked list & \\
    & \\
    \hline
    & \\
    doubly linked list & \\
    & \\
    \hline
    & \\
    binary tree & \\
    & \\
    \hline
    & \\
    k-ary tree & \\
    & \\
    \hline
    & \\
    undirected graph & \\
    & \\
    \hline
    & \\
    directed acyclic graph & \\
    & \\
    \hline
  \end{tabular}
\end{center}

\clearpage

\question
\label{q:container-structures-b}
In question~\ref{q:container-structures-a}, three of the types listed in the table are missing from figures.
Draw a diagram for each missing type.
Clearly indicate which diagram is for which container type.
Make each diagram as similar as possible to one of the figures from question~\ref{q:container-structures-a}.

\end{document}
