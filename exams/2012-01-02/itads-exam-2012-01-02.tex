\documentclass[a4paper]{article}

\usepackage[pdftex]{graphicx}
\usepackage[margin=3cm]{geometry}
\usepackage{verbatim,moreverb,amssymb,amsmath}


\newcounter{question}
\newcommand{\question}{\refstepcounter{question}\section*{Question~\thequestion}}
\renewcommand*\thequestion{\arabic{question}}


\begin{document}

\pagestyle{empty}
\thispagestyle{empty}



\noindent
\begin{minipage}{\columnwidth}
  \centering
  \Large
  DA4002 (HT11) Halmstad University\\
  Introduction to Algorithms, Data Structures, and Problem Solving\\[3\baselineskip]
  \Huge
  Written Exam (Repeat)\\
  \Large
  Monday, January 2, 2012\\
  \emph{exact time and location to be defined}\\[2\baselineskip]
  Examiner: Roland Philippsen
\end{minipage}

\vfill

\noindent
\begin{center}
\fbox{
  \begin{minipage}{0.8\columnwidth}
    \textbf{Student Name:}\\[3\baselineskip]
  \end{minipage}
}
\end{center}

\vfill



\section*{Rules}

Aside from the obvious rules of conduct exams (e.g.\ no chatting):

\begin{itemize}
\item
  \textbf{No computing devices} (laptops, phones, calculators, \emph{etc}).
\item
  \textbf{No books or printouts}.
\item
  \textbf{Allowed self-written notes}: two sheets of A4 paper (front and back).
\end{itemize}



\section*{General Guidelines}

\begin{itemize}
\item
  \textbf{Read carefully} and pace yourself.
  You can solve the problems in any order you want, but later problems may be easier to solve after you have answered the preceding questions.
\item
  \textbf{Write clearly} and draw clear diagrams.
  If you need to correct a mistake, then cleanly cross out the wrong answer and clearly indicate where the correction can be found.
\item
  \textbf{Indicate the question number} for each of your answers.
  If a question has sub-questions, indicate the sub-question number after the main question number, separated by a dot.
  For example, question 3 has 4 sub-questions, and their answers should be numbered 3.1, 3.2, 3.3, and 3.4.
\end{itemize}



\pagebreak
\pagestyle{plain}
\thispagestyle{plain}
\setcounter{page}{1}



\question

For each of the following data storage examples, choose the most appropriate data structure types from the list in the box.
Then draw a small diagram to illustrate the chosen data structure.

\begin{center}
  \fbox{%
    \begin{minipage}{0.7\columnwidth}
      List of data structures to choose from:
      \begin{itemize}
      \item undirected graph
      \item binary search tree
      \item linked list
      \item directed acyclic graph
      \item k-ary tree
      \end{itemize}
    \end{minipage}%
  }
\end{center}


\begin{enumerate}
\item
  Storing a log of meteorological data, such as temperature, atmospheric pressure, wind speed and direction.
  The purpose of the log is to later display graphs of the evolution over time of these data.
\item
  Keeping an inventory of electronical components in a workshop, such as various resistors, capacitors, transistors, etc.
  The purpose is to look up components by part number to see how many are in stock.
\item
  Tracking the prerequisite dependencies between courses at a University.
  The purpose is to help students plan their curriculum and allow professors to check for inconsistencies in study programmes.
\item
  Storing information about animals and plants that are on display in a museum of natural history.
  The purpose is to present visitors with an interface where they can explore the collection according to biological classification.
  For example, animal are grouped into species, which are parts of a genus, which are grouped into families.
\end{enumerate}

\clearpage

\question

Determine the Big-Oh running time complexity for each of the methods whose name begins with \texttt{question\_} in the code below.

\noindent
\begin{minipage}{0.95\columnwidth}
  \footnotesize
  \verbatimtabinput{Q2.java}
\end{minipage}



\end{document}
