\documentclass[a4paper]{article}

\usepackage[pdftex]{graphicx}
\usepackage[margin=3cm]{geometry}
\usepackage{verbatim,moreverb,amssymb,amsmath,paralist}


\newcounter{question}
\newcommand{\question}[1]{\refstepcounter{question}\section*{Question~\thequestion~~~\small\emph{(#1)}}}
\renewcommand*\thequestion{\arabic{question}}


\begin{document}

\pagestyle{empty}
\thispagestyle{empty}



\noindent
\begin{minipage}{\columnwidth}
  \centering
  \Large
  DA4002 (HT13) Halmstad University\\
  Introduction to Algorithms, Data Structures, and Problem Solving\\[3\baselineskip]
  \Huge
  Written Exam\\
  \Large
  Friday, August 15, 2014\\[2\baselineskip]
  Examiner: Roland Philippsen
\end{minipage}

\vfill

\noindent
\begin{center}
\fbox{
  \begin{minipage}{0.8\columnwidth}
    \textbf{Student Name:}\\[3\baselineskip]
  \end{minipage}
}
\end{center}

\vfill



\section*{Rules}

Aside from the obvious rules of conduct exams (e.g.\ no chatting):

\begin{itemize}
\item
  \textbf{No computing devices} (laptops, phones, calculators, \emph{etc}).
\item
  \textbf{No books or printouts} except for non-electronic dictionaries.
\item
  \textbf{Allowed hand-written notes}: two sheets of A4 paper (front and back).
\end{itemize}



\section*{Guidelines}

\begin{itemize}
\item
  \textbf{Read carefully} and pace yourself.
  Some questions include explanations or hints.
\item
  \textbf{Write clearly} and draw clear diagrams.
\item
  \textbf{Indicate the question number} for each of your answers.
  If a question has sub-questions, clearly indicate which one your answer belongs to.
\item
  \textbf{Do not write with a red pen.}
\end{itemize}



\section*{Evaluation Criteria}

You need 60\% of points to pass.
The remaining 40\% are equally divided among the grades.

\begin{itemize}
\item
  Halmstad: \quad
  5: $\left[87,100\right]\%$ \quad
  4: $\left[74,87\right)\%$ \quad
  3: $\left[60,74\right)\%$
\item
  ECTS: \quad
  A: $\left[92,100\right]\%$ \quad
  B: $\left[84,92\right)\%$ \quad
  C: $\left[76,84\right)\%$ \quad
  D: $\left[68,76\right)\%$ \quad
  E: $\left[60,68\right)\%$
\end{itemize}



\pagebreak
\pagestyle{plain}
\thispagestyle{plain}
\setcounter{page}{1}



\question{4 points}

Consider the following four pseudo-code fragments A, B, C, and D:\\

\small
\noindent
\begin{minipage}[t]{0.44\columnwidth}
  
  \begin{tabbing}
    \quad\=\quad\=\quad\=\quad\=\quad\=\kill
    function \textbf{fragmentA} (\emph{arr}, \emph{len}):\\
    \>// \emph{arr} is an array of numbers\\
    \>// \emph{len} is the length of the array\\
    \>for \emph{i} in 0\ldots\emph{len}-1:\\
    \>\>for \emph{j} in 1\ldots\emph{len}-1:\\
    \>\>\>if \emph{arr}[\emph{j}] $<$ \emph{arr}[\emph{j-1}] then\\
    \>\>\>\>swap \emph{arr}[\emph{j}] with \emph{arr}[\emph{j-1}]\\
    \>\>\>end if\\
    \>\>end for\\
    \>end for\\
    end function
  \end{tabbing}
  
  \begin{tabbing}
    \quad\=\quad\=\quad\=\quad\=\quad\=\kill
    function \textbf{fragmentC} (\emph{start})\\
    \>// - \emph{start} is a node\\
    \>// variables:\\
    \>// - \emph{open} is a queue of nodes\\
    \>// - \emph{closed} is a set of nodes\\
    \>// - \emph{it} and \emph{nb} are nodes\\
    \>\emph{open}.push(\emph{start})\\
    \>while \emph{open} is not empty:\\
    \>\>\emph{it} = \emph{open}.pop()\\
    \>\>if \emph{closed} does not contain \emph{it} then\\
    \>\>\>\emph{closed}.add(\emph{it})\\
    \>\>\>for \emph{nb} in \emph{it}.neighbors:\\
    \>\>\>\>if \emph{closed} does not contain \emph{nb} then\\
    \>\>\>\>\>\emph{nb}.predecessor = \emph{it}\\
    \>\>\>\>\>\emph{open}.push(\emph{nb})\\
    \>\>\>\>end if\\
    \>\>\>end for\\
    \>\>end if\\
    \>end while\\
    end function
  \end{tabbing}
  
\end{minipage}%
\hfill%
\begin{minipage}[t]{0.44\columnwidth}
  
  \begin{tabbing}
    \quad\=\quad\=\quad\=\quad\=\quad\=\kill
    function \textbf{fragmentB} (\emph{arr}, \emph{len}, \emph{num}):\\
    \>// - \emph{arr} is an array of numbers\\
    \>// - \emph{len} is the length of the array\\
    \>// - \emph{num} is a number\\
    \>for \emph{i} in 0\ldots\emph{len}-1:\\
    \>\>if \emph{arr}[\emph{i}] $=$ \emph{num} then\\
    \>\>\>return \emph{i}\\
    \>\>end if\\
    \>end for\\
    \>return -1\\
    end function
  \end{tabbing}
  
  \begin{tabbing}
    \quad\=\quad\=\quad\=\quad\=\quad\=\kill
    function \textbf{fragmentD} (\emph{start})\\
    \>// - \emph{start} is a node\\
    \>// NOTE:\\
    \>// \quad someFunction() can be assumed to be given.\\
    \>// \quad It does not matter what that function does,\\
    \>// \quad and it does not modify its argument.\\
    \>someFunction(\emph{start})\\
    \>if \emph{start}.left is not null then\\
    \>\>fragmentD(\emph{start}.left)\\
    \>end if\\
    \>if \emph{start}.right is not null then\\
    \>\>fragmentD(\emph{start}.right)\\
    \>end if\\
    end function
  \end{tabbing}
\end{minipage}
\normalsize
\vspace{\baselineskip}

For each of those code fragments, choose which algorithm name from the table below best corresponds to what the fragment does, and mark its letter \emph{(A, B, C, or D)} in the space on the right.

\begin{center}
  \begin{tabular}{|l|p{0.3\columnwidth}|}
    \hline
    \emph{algorithm} & \emph{A, B, C, or D} \\
    \hline
    linear search & \\
    \hline
    binary search & \\
    \hline
    breadth-first search & \\
    \hline
    depth-first search & \\
    \hline
    pre-order traversal & \\
    \hline
    level-order traversal & \\
    \hline
    heap sort & \\
    \hline
    bubble sort & \\
    \hline
  \end{tabular}
\end{center}


\clearpage

\question{4 points}

For each of the following code fragments,
\begin{itemize}
\item
  develop a formula allowing to estimate the running time depending on the parameter $N$,
\item
  and then determine its Big-Oh runtime complexity.
\end{itemize}

\noindent
Note that the method \texttt{f\_1ms} always requires 1 ms to run, and the method \texttt{f\_3ms} always takes 3 ms.
All other operation delays can be neglected.

\begin{itemize}
\item
  fragment 1:\\
  \begin{minipage}[t]{0.4\columnwidth}
    \footnotesize
    \verbatimtabinput{q2a.txt}
  \end{minipage}
\item
  fragment 2:\\
  \begin{minipage}[t]{0.4\columnwidth}
    \footnotesize
    \verbatimtabinput{q2b.txt}
  \end{minipage}
\item
  fragment 3:
  \emph{(important: compute the runtime of the \texttt{for} loop, not just the \texttt{foo} function)}\\
  \begin{minipage}[t]{0.4\columnwidth}
    \footnotesize
    \verbatimtabinput{q2c.txt}
  \end{minipage}
\item
  fragment 4:\\
  \begin{minipage}[t]{0.4\columnwidth}
    \footnotesize
    \verbatimtabinput{q2d.txt}
  \end{minipage}
\end{itemize}


\clearpage

\question{6 points}

\begin{enumerate}

\item
  The execution time $T_A$ of program $A$ has been measured on a problem of size $N=1024$:
  it took $T_A(N) = 640ms$ to run.
  What is the expected execution time $T'_A$ for a problem of size $N'=128$, if $A$'s big-Oh complexity is\\
  \emph{(i)} logarithmic: $T_A(N) \in O(\log N)$ ?\\
  \emph{(ii)} square: $T_A(N) \in O(N^2)$ ?
  
  \emph{\textbf{Notice the table of numeric values} for $x^2$, $x^3$, and $2^x$ on the bottom of this page, to help you with the computations.}
  
\item
  Program $B$ required $T_B=10ms$ to run on a problem of size $N=1000$.
  What problem size $N'$ can be expected to complete, if the available time is $T'_B=270ms$ and the big-Oh complexity of $B$ is\\
  \emph{(i)} linear: $T_B(N) \in O(N)$ ?\\
  \emph{(ii)} cubic: $T_B(N) \in O(N^3)$ ?
  
\item
  The time $T_C$ it takes to run \texttt{function\_C} for $N=512$ is $T_C=5ms$.
  Based on a big-Oh complexity analysis of the code shown below, how much time will it take for $N'=2048$?
  You can assume that the various \texttt{slow\_X} functions take significantly longer to execute than any of the other operations.
  
\item
  The time $T_D$ taken by \texttt{function\_D} given below is $30ms$ for an array of length $N=512$.
  How large of an array can be handled when we only have $10ms$ available?
  Again, you can assume that the \texttt{slow\_5} function takes up the majority of the computation time.
  
\end{enumerate}
  
\vfill

\noindent
\fbox{%
  \begin{minipage}[t]{0.40\columnwidth}
    \small
    \verbatimtabinput{q3a.txt}
\end{minipage}}\hfill
\fbox{%
  \begin{minipage}[t]{0.56\columnwidth}
    \small
    \verbatimtabinput{q3b.txt}
\end{minipage}}

\vfill

\noindent
\begin{tabular}{|l|r|r|r|r|r|r|r|r|r|r|r|r|r|r|r|}
  \hline
  $x=$   & 0 & 1 & 2 &  3 &  4 &   5 &   6 &   7 &   8 &   9 &   10 &   11 &   12 &   13 &    14 \\
  $x^2=$ & 0 & 1 & 4 &  9 & 16 &  25 &  36 &  49 &  64 &  81 &  100 &  121 &  144 &  169 &   196 \\
  $x^3=$ & 0 & 1 & 8 & 27 & 64 & 125 & 216 & 343 & 512 & 729 & 1000 & 1331 & 1728 & 2197 &  2744 \\
  $2^x=$ & 1 & 2 & 4 &  8 & 16 &  32 &  64 & 128 & 256 & 512 & 1024 & 2048 & 4096 & 8192 & 16384 \\
  \hline
\end{tabular}


\clearpage

\question{6 points}

% 6 pts for the second table (1 pt per correct row, but be nice concerning ``Folgefehler'')

Dynamic programming can be used to choose coins that add up to a given amount, such that the number of coins is minimal.
The following table shows this when the available coins are $C=\{1, 4\}$, for total amounts of up to $K=6$.

The \emph{value} $F(K)$ stands for the minimum amount of coins to create a total of $K$.
At each step, it is the minimum of the available options $f_1(K)$ and $f_4(K)$.
Thus, $F$ is computed using $F(K)=\min(f_1(K), f_4(K))$.
If an option is not available for a given step, it is marked $\varnothing$.
Notice that the table starts at $K=0$, where $F(0)=0$.

\begin{center}
  \begin{tabular}{|c|l|l|l|c|}
    \hline
    \textbf{step} &
    \multicolumn{3}{|c|}{\textbf{choices}} &
    \textbf{value}
    \\
    $K$ &
    cost of $c=1$ &
    cost of $c=4$ &
    \textbf{best} $c$ &
    $F$ \\
    \hline
    $0$ &
    $\varnothing$ &
    $\varnothing$ &
    $\varnothing$ &
    $F(0) = 0$
    \\
    \hline
    $1$ &
    $1+F(0)=1$ &
    $\varnothing$ &
    $c=1$ &
    $F(1) = 1$
    \\
    \hline
    $2$ &
    $1+F(1)=2$ &
    $\varnothing$ &
    $c=1$ &
    $F(2) = 2$
    \\
    \hline
    $3$ &
    $1+F(2)=3$ &
    $\varnothing$ &
    $c=1$ &
    $F(3) = 3$
    \\
    \hline
    $4$ &
    $1+F(3)=4$ &
    $1+F(0)=1$ &
    $c=4$ &
    $F(4) = 1$
    \\
    \hline
    $5$ &
    $1+F(4)=2$ &
    $1+F(1)=2$ &
    $c\in\{1,4\}$ &
    $F(5) = 2$
    \\
    \hline
    $6$ &
    $1+F(5)=3$ &
    $1+F(2)=3$ &
    $c\in\{1,4\}$ &
    $F(6) = 3$
    \\
    \hline
  \end{tabular}
\end{center}

\vfill

Fill in the table below, similar to the one above, for up to $K=6$ \textbf{when there also is a $c=3$ coin}.
In other words, use $C=\{1, 3, 4\}$, add $f_3(K)$ to the available choices, and use $F=\min(f_1, f_3, f_4)$.

\vfill

\begin{center}
  \begin{tabular}{|c|p{0.22\columnwidth}|p{0.22\columnwidth}|p{0.22\columnwidth}|l|c|}
    \hline
    \textbf{step} &
    \multicolumn{4}{|c|}{\textbf{choices}} &
    \textbf{value}
    \\
    $K$ &
    cost of $c=1$ &
    cost of $c=3$ &
    cost of $c=4$ &
    \textbf{best} $c$ &
    $F$ \\
    \hline
    $0$ &
    $\varnothing$ &
    $\varnothing$ &
    $\varnothing$ &
    $\varnothing$ &
    $F(0) = 0$
    \\
    \hline
    $1$ & & & & & \\ & & & & & \\ & & & & & \\ & & & & & \\
    \hline
    $2$ & & & & & \\ & & & & & \\ & & & & & \\ & & & & & \\
    \hline
    $3$ & & & & & \\ & & & & & \\ & & & & & \\ & & & & & \\
    \hline
    $4$ & & & & & \\ & & & & & \\ & & & & & \\ & & & & & \\
    \hline
    $5$ & & & & & \\ & & & & & \\ & & & & & \\ & & & & & \\
    \hline
    $6$ & & & & & \\ & & & & & \\ & & & & & \\ & & & & & \\
    \hline
  \end{tabular}
\end{center}


\clearpage

\question{6 points}

Imagine that your are working as a programmer and encounter the following situations.

\begin{enumerate}
\item
  One of your colleagues wants to speed up an existing program that manages information about persons.
  The program currently keeps the information in a linked list.
  This list is only rarely modified, but very frequently used to search for persons.
  Your colleague plans to change the program so that it sorts the information by name before storing it in the linked list, and then uses binary search.
  Do you think this is a good idea?
  Why, or why not?
\item
  Another colleague, working on a similar but different program, has written functions for saving the data from a binary search tree in a file and later loading them back from the file.
  He implemented the safe function using in-order traversal, and everything worked correctly and quickly during his tests with small data sets.
  However, with large data sets, the program becomes very slow after loading data from such a file.
  What do you think is going wrong?
  What do you suggest to solve this problem?
\item
  You are leading a team that works on designing and implementing an email routing system that is expected to handle a very large amount of messages.
  The idea is to use up to 10 priority levels so that important email gets sent first while less important messages wait.
  Your team has come up with three suggestions for implementing the message queue:
  \begin{compactitem}
  \item
    use a min heap with array-backed storage;
  \item
    use an array of 10 linked lists, one for each priority level;
  \item
    use a binary search tree.
  \end{compactitem}
  Which of these solutions would you choose, and why?
\end{enumerate}


\end{document}
