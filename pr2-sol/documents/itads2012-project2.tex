\documentclass[a4paper,10pt]{article}

\usepackage[margin=3cm]{geometry}
\usepackage{verbatim,url}

\begin{document}

\title{
  {\normalsize
    Introduction to Algorithms, Data Structures, and Problem Solving\\
    DA4002 (HT12) Halmstad University}\\
  Assignment for Project 2: Sequence Alignment\\
}
\author{
  Roland Philippsen\\
  \texttt{roland.philippsen@hh.se}
}
\maketitle



\section{Introduction}

Finding a good alignment between two sequences of symbols (characters) is a commonly encountered problem in.
For example, it can be used to compute differneces between files, as part of spell-checking programs, and especially in bioinformatics to align protein or nucleotide sequences.
A prominent example is the Needleman-Wunsch algorithm~\cite{needleman-wunsch:1970}.

In this project, you will implement the Needleman-Wunsch algorithm, based on the slides of lecture 6~\cite{lecture6} and the excellent article on Wikipedia~\cite{wikipedia:needleman-wunsch}.
You can also search for more information about the algorithm on the internet, of course.
As a starting point for the implementation, you are provided with example code for two-dimensional arrays in C, hints on dealing with character data, and example outputs from a sequence-alignment program working on various input strings.
Of course, all lecture and exercise material, the course books, and the work you did for project~1 can be helpful, too.

The deadline for handing in the source code and the report is \textbf{Friday, October 19, 2012, at 18h00}.
Teams who miss the deadline will receive a penalty of 5 points (the maximum number of points is 25).
In case of exonerating circumstances, such as sickness certified by a medical doctor, a deadline extension will be granted.
Participants must notify the lecturer of such circumstances as soon as possible when they arise.



\section{Starting Point}

they can do their own makefile\\
will give them some examples:\\
-- matrix using 1d array\\
-- matrix using array of arrays\\
-- printing numbers and strings right-aligned into fixed width\\
-- simple example of propagating values through a matrix and printing some properties\\
-- reading two strings from the command line and checking that they only contain letters\\



\section{Assignment}

-- M implement computation of match cost (can be a function or a similarity matrix)\\
-- M fill value array based on match, insert, delete\\
-- M give ``just'' the alignment score for two strings\\
-- M print the value matrix\\
-- M write a report: what file implements what, how to compile, how to run, example output for some example pairs\\
-- O print the value matrix with backpointer information\\
-- O trace back one optimal alignment\\
-- O trace back all optimal alignments\\
-- O implement Smith-Waterman algorithm and compare results with Needleman-Wunsch\\

A properly performed set of mandatory tasks that are well documented in the project report is worth 16 points (HH grade 4, ECTS grade D).
Bonus tasks can give up to 9 extra points, such that the maximum achievable number of points is 25 (HH grade 5, ECTS grade A).

Keep in mind that writing the report is an integral part of the project, so do not spend too much time on extending the functionality.
It is important to note that teams are expected to manage their resources by themselves.
This includes apportioning the time available for finding information online and in the course books, developing and debugging code, documenting the work, and preparing and testing the project archive file that you submit for evaluation.
How these aspects are shared between the team members is for each team to decide.

\subsection{Mandatory Tasks}

\begin{enumerate}

\item
  one

\end{enumerate}



\section{Bonus Tasks}

\begin{enumerate}

\item
  one
  
\end{enumerate}



\section{Further Information}

\subsection*{Preparing Archives for Submitting Your Project}

You already encountered the \texttt{tar} program in exercise 7 to extract project archives.
In order to create an archive, use the command ``\texttt{tar xfvj archive-name.tar.bz2 project-directory}'' where you have to replace \texttt{project-directory} with the name of your project directory, and the result will be in \texttt{archive-name.tar.bz2}.

However, please don't choose just any name.
Rather, make a separate directory based on the family names of the team members, copy all the source and data files there, then create the archive with an archive name that has the directory name as its base.

For example, two students named ``Alice Foo'' and ``Bob Bar'' would proceed as follows:
\begin{enumerate}
\item
  Create a directory based on student and project names.\\
  For example: ``\texttt{mkdir itads-proj2-foo-bar}''
\item
  Copy all the relevant files there.\\
  For example: ``\texttt{cp *.h *.c Makefile itads-proj2-foo-bar/.}''
\item
  Create the archive: ``\texttt{tar cfvj itads-proj2-foo-bar.tar.bz2 itads-proj2-foo-bar}''
\item
  \textbf{IMPORTANT!}
  Double-check that the files you placed in the archive are complete and free of errors.
  This is best done by going into the directory which you just used to create the archive, and trying to build and run your applications.
\end{enumerate}



\footnotesize
\bibliographystyle{plain}
\bibliography{itads-bibliography}

\end{document}
